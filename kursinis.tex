\documentclass{VUMIFPSkursinis}
\usepackage{algorithmicx}
\usepackage{algorithm}
\usepackage{algpseudocode}
\usepackage{amsfonts}
\usepackage{amsmath}
\usepackage{bm}
\usepackage{caption}
\usepackage{color}
\usepackage{float}
\usepackage{graphicx}
\usepackage{listings}
\usepackage{subfig}
\usepackage{wrapfig}
\usepackage{titlecaps}
% titlecaps includes \titlecap command which makes titlecase
% Below command specifies words which shouldn't be capitalized in title case
\Addlcwords{is of on in the a an for and}

\usepackage{enumitem}
%PAKEISTA, tarpai tarp sąrašo elementų
\setitemize{noitemsep,topsep=0pt,parsep=0pt,partopsep=0pt}
\setenumerate{noitemsep,topsep=0pt,parsep=0pt,partopsep=0pt}

% Titulinio aprašas
\university{Vilniaus universitetas}
\faculty{Faculty of Mathematics and Informatics}
\department{Software engineering bachelor study programme}
\papertype{Project work}
\title{\titlecap{Tema in title case english}}
%\titleineng{Investigation Methods of Software Development}
\status{4th year, 5th group student}
\author{Vardenis Pavardenis}
% \secondauthor{Vardonis Pavardonis}   % Pridėti antrą autorių
% \thirdauthor{Vardonis Pavardonis}   % Pridėti trečią autorių
% \fourthauthor{Vardonis Pavardonis}   % Pridėti ketvirtą autorių
%\version{0.1}
\supervisor{Assoc. Prof. Vardonis Pavardonis}
\date{Vilnius – \the\year}

% Nustatymai
% \setmainfont{Palemonas}   % Pakeisti teksto šriftą į Palemonas (turi būti įdiegtas sistemoje)
\bibliography{bibliografija}

\begin{document}
	
% PAKEISTA	
\maketitle
\cleardoublepage\pagenumbering{arabic}
\setcounter{page}{2}

%TURINYS
\tableofcontents

\sectionnonum{Introduction}
% Įvade apibūdinamas darbo tikslas, temos aktualumas ir siekiami rezultatai.
% Darbo įvadas neturi būti dėstymo santrauka. Įvado apimtis 1–2 puslapiai.

\section{Interoperability methods}
\subsection{Notary schemes}

\subsection{Relays}

\subsection{Hash-locking}

\sectionnonum{Results and conclusion}
Rezultatų ir išvadų dalyje turi būti aiškiai išdėstomi pagrindiniai darbo
rezultatai (kažkas išanalizuota, kažkas sukurta, kažkas įdiegta) ir pateikiamos
išvados (daromi nagrinėtų problemų sprendimo metodų palyginimai, teikiamos
rekomendacijos, akcentuojamos naujovės).


%% PAKEISTAS PAVADINIMAS Į 'Šaltiniai'
\printbibliography[heading=bibintoc, title=References]  % Šaltinių sąraše nurodoma panaudota
% literatūra, kitokie šaltiniai. Abėcėlės tvarka išdėstomi darbe panaudotų
% (cituotų, perfrazuotų ar bent paminėtų) mokslo leidinių, kitokių publikacijų
% bibliografiniai aprašai.  Šaltinių sąrašas spausdinamas iš naujo puslapio.
% Aprašai pateikiami netransliteruoti. Šaltinių sąraše negali būti tokių
% šaltinių, kurie nebuvo paminėti tekste.

% \sectionnonum{Sąvokų apibrėžimai}

\appendix  % Priedai
% Prieduose gali būti pateikiama pagalbinė, ypač darbo autoriaus savarankiškai
% parengta, medžiaga. Savarankiški priedai gali būti pateikiami ir
% kompaktiniame diske. Priedai taip pat numeruojami ir vadinami. Darbo tekstas
% su priedais susiejamas nuorodomis.

\section{Implementation of a portable token using signature-locking}
Main parts of this appendix:
\begin{itemize}
  \item AAA
  \item BBB
\end{itemize}


\end{document}
